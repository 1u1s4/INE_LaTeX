\documentclass[11pt,twoside]{book}
%Versión 4.0
% paquetes-----------------------------------------------------------------------------------------
\usepackage[
paperwidth = 8.5in,
paperheight = 11in,
left = 1.25in,
right = 0.75in,
top = 0.950in,
bottom = 0.925in
]{geometry}
\usepackage{amsmath}
\usepackage{amsfonts}
\usepackage{amssymb}
\usepackage{graphicx}
\usepackage{pdfpages}
\usepackage{setspace} 
\usepackage{xltxtra}
\usepackage{enumitem}
\usepackage{fixltx2e}
\usepackage{xifthen}
\usepackage{xargs}
\usepackage{booktabs}
\usepackage{multirow}
\usepackage{lipsum} % para pruebas
\usepackage{pdflscape}
\usepackage{rotating}
\usepackage{bigstrut}
\usepackage{longtable}
\usepackage{fancyhdr}
\usepackage{url}
% aqui sigue
% Tabla de contenidos y vinculaciones
\usepackage{tocloft}
\usepackage[hidelinks, verbose]{hyperref}
% Elementos geométricos de diseño del cuerpo (cajas de colores, etc.)
\usepackage{colortbl}
\usepackage{multicol}
\setlength{\columnsep}{1.1cm}
% Cambios de márgenes y según paridad de hojas
\usepackage{changepage}
\strictpagecheck
% Paquete Tcolorbox
\usepackage[skins, breakable, hooks]{tcolorbox}
\usepackage[input-decimal-markers={.}, input-ignore={,}, group-separator={,}]{siunitx}
% Para compilar en XeLaTeX con tildes
\usepackage{polyglossia}
\setmainlanguage{spanish}
\usepackage{tikz}
\usetikzlibrary{calc}
\usetikzlibrary{positioning}
\usepackage{array}
% leer pag 4 https://mirrors.ucr.ac.cr/CTAN/macros/latex/required/tools/array.pdf
% nuevos tipos de columnas
\newcolumntype{x}[1]{%
	>{\centering\arraybackslash}p{#1}}%
\newcolumntype{g}[1]{%
	>{\raggedleft\arraybackslash}p{#1}}%
\newcolumntype{q}[1]{%
	>{\raggedright\arraybackslash}p{#1}}%
% Tipo de letra
% \usepackage{fontspec} % en teoria esto se carga en xltxtra
\setmainfont[
BoldFont = OpenSans-CondBold.ttf ,
ItalicFont = OpenSans-CondLightItalic.ttf ,
BoldItalicFont = OpenSans-CondLightItalic.ttf ]{OpenSans-CondLight.ttf}
\newfontfamily\Bold{Open Sans Condensed Bold}
\newfontfamily\Sans{Open Sans}
\newfontfamily\SansBold{Open Sans Bold}
\newfontfamily\Italic{Open Sans Condensed Light Italic}
\newfontfamily\Logos{Latin Modern Roman}
\newfontfamily\Cinzel{Cinzel}
% FIN-paquetes-------------------------------------------------------------------------------------

% Estilo de página en blanco en el índice
\newcommand{\blank}{\addtocontents{toc}{\protect\thispagestyle{empty}}}

%título y fuente
\newcommand{\titulodoc}{Nombre del documento}
\newcommand{\lafuente}{Estadísticas INE}

\newcounter{Cuadro}[chapter]
\renewcommand{\theCuadro}{\thechapter.\arabic{Cuadro}}

\newcommand{\titulocuadro}[1]{\addtocounter{Cuadro}{1}
{\Bold\color{color1!80!black}{\normalsize Cuadro \theCuadro $\,-$  #1 }}
}

%%%%%%%%%%% Diseño global del documento
%	\setlength{\headsep}{0pt}
%	\setlength{\footskip}{46pt}
\setlength{\parindent}{1.5em}		%sangría
\setlength{\parskip}{2ex}		%separación entre párrafos  

%Distancias
\newlength{\cuadri} 
\setlength{\cuadri}{0.125in}

%Formato de contenidos
\setlength{\cftbeforetoctitleskip}{0em}
\AtBeginDocument{\addtocontents{toc}{\protect\thispagestyle{empty}}} 

\makeatletter
\renewcommand*\l@subsection{\@dottedtocline{2}{5.2em}{3.2em}}
\makeatother

\renewcommand{\thesection}{\thechapter.\arabic{section}}

\cftsetpnumwidth{2\cuadri}
\cftsetrmarg{8\cuadri}
\renewcommand{\cftsecnumwidth}{2.5\cuadri}
\renewcommand{\cftchapnumwidth}{2\cuadri}
\renewcommand{\cftsecindent}{2\cuadri}

%Colores base del documento
\definecolor{color1}{rgb}{0,0,1}
\definecolor{color2}{rgb}{0.3,0.5,1}

%Para que las páginas en blanco no estén numeradas
\let\origdoublepage\cleardoublepage
\newcommand{\clearemptydoublepage}{
	\clearpage
	{\pagestyle{empty}\origdoublepage}}
\let\cleardoublepage\clearemptydoublepage

%%%%%%%% Llamadas y notas al pie
\makeatletter
\newcommand{\markerspace}{\@ifnextchar.%
{$\!$}{\@ifnextchar,%
    {$\!$}{\@ifnextchar;%
        {$\!$}{\@ifnextchar:%
            {$\!$}{$\ $}
        }
    }
}}
\makeatother

\newcounter{numllamada}
\newcounter{numtextollamada}
\setcounter{numllamada}{0}
\setcounter{numtextollamada}{0}

\newcommand{\llamada}[1][\thenumllamada]{
	\stepcounter{numllamada}
	\begingroup
	\setcounter{mpfootnote}{#1}
	\renewcommand\thefootnote\thempfootnote$\hspace{0.2ex}$\footnotemark
	\endgroup
	\markerspace
}

\newcommand{\notita}[2][\thenumtextollamada]{
	\stepcounter{numtextollamada}
	\stepcounter{numllamada}
	$\hspace{0.2ex}$
	\footnote[#1]{#2}
}

\newcommand{\textollamada}[2][\thenumtextollamada]{%
\ifthenelse{\equal{#1}{*}}{
	\begingroup
	\renewcommand\thefootnote\thempfootnote\footnotetext[0]{#2\\[-1.7ex]}
	\endgroup
	}{
	\stepcounter{numtextollamada}
	\begingroup
	\renewcommand\thefootnote\thempfootnote\footnotetext[#1]{#2}
	\endgroup
	}
}

%redefinición de footnote
\newlength{\footnoterulewidth}
\setlength{\footnoterulewidth}{2.5cm}
\newlength{\footnoteruleheight}
\setlength{\footnoteruleheight}{.4pt} 
\makeatletter
\renewcommand{\footnoterule}{
	\kern -3pt
	\color{color2}
	\hrule width 
	\footnoterulewidth height 
	\footnoteruleheight
	\kern
	\dimexpr 3pt - \footnoteruleheight \relax
} 
\makeatother

\makeatletter
\renewcommand\@makefntext[1]{
	\noindent
	\makebox[1em][r]{\scriptsize\@makefnmark}
	\scriptsize#1
}
\makeatother

%%%%%%%%%%% tablas
\LTcapwidth=1.234\textwidth
\setlength{\arrayrulewidth}{0.8pt}
\arrayrulecolor{color2}

% Definición del comando cajita

\newtcolorbox{cajita-arriba}{
	width = 52\cuadri,
	height = 35\cuadri,
	enlarge left by = 0pt,
	enlarge top by = 1\cuadri,
	enlarge bottom by = 2\cuadri,
	nobeforeafter,
	colframe = white,
	colback = white,
	left = -3pt,
	right = 0pt,
	bottom = 0pt,
	top = -3pt,
	arc = 0pt,
	boxrule = 0pt
}

\newtcolorbox{cajita-abajo}{
	width = 52\cuadri,
	height = 35\cuadri,
	enlarge top by = 0pt,
	enlarge left by = 0pt,
	enlarge bottom by = -2\cuadri,
	nobeforeafter,
	colframe = white,
	colback = white,
	left = -3pt,
	right = 0pt,
	bottom = 0pt,
	top = -3pt,
	arc = 0pt,
	boxrule = 0pt
}

\newtcolorbox{cajota-unica}{
	width = 52\cuadri,
	height = 72\cuadri,
	enlarge left by = 0pt,
	enlarge top by  = 1\cuadri,
	enlarge bottom by = -2\cuadri,
	nobeforeafter,
	colframe = white,
	colback = white,
	left = -3pt,
	right = 0pt,
	bottom =  0pt,
	top = -3pt,
	arc = 0pt,
	boxrule = 0pt
}

\newtcolorbox{descripcion-cajita}{
	width = 18\cuadri,
	height = 30.8\cuadri,
	nobeforeafter,
	colframe = white,
	colback = white,
	left = -3pt,
	right = -3pt,
	bottom = -3pt,
	top = -8pt,
	arc = 0pt,
	boxrule = 0pt,
	enlarge bottom by = -29.78\cuadri
}

\newtcolorbox{grafica-cajita}{
	width = 32\cuadri,
	height = 30.8\cuadri,
	nobeforeafter,
	colframe = white,
	colback = white,
	left = -3pt,
	right = -3pt,
	bottom = -3pt,
	top = -2pt,
	arc = 0pt,
	boxrule = 0pt,
	enlarge bottom by = -29.78\cuadri
}
% Cajas para fondos de capítulo y encabezado
\newtcolorbox{fondo-capitulo}{
	width = 8.5in,
	height = 11in,
	skin = enhancedmiddle, 
	nobeforeafter,
	watermark graphics = fondo-capitulo.pdf,
	watermark opacity = 1.0,
	watermark overzoom = 1.0,
	enlarge left by = -1.525in,
	enlarge top by = -0.95in,
	enlarge bottom by = -20\cuadri,
	boxrule = 0pt,
	colframe = white,
	left = -3pt,
	bottom = -1pt,
	top = 8\cuadri,
	right = -3pt,
	arc = 0pt
}

\newtcolorbox{parte-toc}{
	width = 52\cuadri,
	enlarge left by = 0pt,
	enlarge top by = 0\cuadri,
	enlarge bottom by = 0\cuadri,
	nobeforeafter,
	colframe = color1!90!black,
	colback = white,
	left = 3pt,
	right = 0pt,
	bottom = 0pt,
	top = 0pt,
	arc = 0pt,
	boxrule = 0pt,
	leftrule = 4pt
}

\newtcolorbox{fondo-parte}{
	width=8.5in,
	height=11in,
	skin=enhancedmiddle,
	nobeforeafter,
	watermark graphics=parte.pdf,
	watermark opacity=1.0,
	watermark overzoom=1.0,
	enlarge left by=-1.455in,
	enlarge top by=-0.95in, 
	enlarge bottom by=-20\cuadri,
	boxrule=0pt,
	colframe=white,
	left=-3pt,
	bottom=-1pt,
	top=8\cuadri,
	right=-3pt,
	arc=0pt
}

\newtcolorbox{fondo-capitulo-no-descripcion}{
	width=8.5in,
	height=11in,
	skin=enhancedmiddle,
	nobeforeafter,
	watermark graphics=fondo-capitulo-no-descripcion.pdf,
	watermark opacity=1.0,
	watermark overzoom=1.0,
	enlarge left by=-1.525in,
	enlarge top by=-0.95in,
	enlarge bottom by=-20\cuadri,
	boxrule=0pt,
	colframe=white,
	left=-3pt,
	bottom=-1pt,
	top=8\cuadri,
	right=-3pt,
	arc=0pt
}

\newtcolorbox{numcapitulo}{
	height=1.2in,
	width=1.12in,
	enlarge top by= -1.3in,
	enlarge bottom by=-0.47in,
	boxrule=3pt,arc=3pt,
	colframe=color1,
	colback=color1,
	right=2pt,
	left=3pt,
	top=14pt,
	bottom=2pt
}

\newtcolorbox{encabezadoimpar}{
	width=8.5in,
	height=0.86in,
	skin=enhancedmiddle,
	nobeforeafter,
	watermark graphics=topodd3.pdf,
	watermark opacity=1.0,
	watermark overzoom=1.0,
	enlarge left by=-1.25in,
	enlarge top by=-0.45in,
	enlarge bottom by=3\cuadri,
	boxrule=0pt, colframe=white,
	left=0.71in,
	bottom=-1pt,
	top=2.5\cuadri,
	right=0.71in,
	arc=0pt
}

\newtcolorbox{encabezadopar}{
	width=8.5in,
	height=0.86in,
	skin=enhancedmiddle,
	nobeforeafter,
	watermark graphics=topeven3.pdf,
	watermark opacity=1.0,
	watermark overzoom=1.0,
	enlarge left by=-0.75in,
	enlarge top by=-0.45in,
	enlarge bottom by=3\cuadri,
	boxrule=0pt,
	colframe=white,
	left=0.71in,
	bottom=-1pt,
	top=2.5\cuadri,
	right=0.71in,
	arc=0pt
}

\newtcbox{numpag}{
	colback=color1,
	colframe=color1,
	arc=0pt,
	top=3pt,
	bottom=2pt,
	left=3pt,
	right=3pt,
	nobeforeafter,
	enlarge top by=-0.125in
}

%%%%%%%%%%% Encabezado y pie de página
\fancypagestyle{estandar}{
	\fancyhf{}
	\renewcommand{\headrulewidth}{0pt}
	\fancyhead[CO]{
		\begin{encabezadoimpar}
		\end{encabezadoimpar}
	}
	\fancyfoot[RO]{
		\color{color2}{\capituloencabezado \ \ }
		\color{black}
		\raisebox{0.5mm}{$\mid$}
		\color{color1}
		\textbf{ \ \ \ \thepage}
	}
	\fancyhead[CE]{
		\begin{encabezadopar}
		\end{encabezadopar}
	}
	\fancyfoot[LE]{
		\color{color1}
		\textbf{\thepage \ \ \ }
		\color{black}
		\raisebox{0.5mm}{$\mid$}
		\color{color2}{ \ \  \titulodoc}
	}
}

\fancypagestyle{soloarriba}{
	\fancyhf{}
	\renewcommand{\headrulewidth}{0pt}
	\fancyhead[CO]{
		\begin{encabezadoimpar}
		\end{encabezadoimpar}
	}
	\fancyfoot[RO]{}
	\fancyhead[CE]{
		\begin{encabezadopar}
		\end{encabezadopar}
	}
	\fancyfoot[LE]{}
}

\pagestyle{estandar}

%%%%%%%%%%% Macros de cajitas
%  El comando se escribe así: \cajita[Sección en índice]{Sección en cuerpo}{Descripción}{Título gráfica}{Desagregación}{Gráfica con \includegraphics o tikz}{Fuente}
\newcounter{updown}
\setcounter{updown}{0}

\newcommand{\cajitaalternante}[1]{
\ifthenelse{\equal{\theupdown}{0}}{
	\noindent
	\begin{cajita-arriba}
		\phantomsection
		\stepcounter{section} #1
	\end{cajita-arriba}
	\setcounter{updown}{1}
	}{
	\noindent
	\begin{cajita-abajo}
		\phantomsection
		\stepcounter{section} #1
	\end{cajita-abajo}
	\setcounter{updown}{0}
	}
}

\newcommand{\titulizador}[1]{
	\begin{tabular}{@{}p{3.5\cuadri}|p{3pt}@{}p{46.0\cuadri}}
		& &\\[-1\cuadri]
		\textbf{\color{color2}\Large \thesection}$\ $ & & \textbf{\Large #1}\\[-1\cuadri]
		& &\\[-0.8pt] \cline{2-3}
	\end{tabular}
}

\newcommand{\titulizadormanual}[2]{
	\begin{tabular}{@{}p{3.5\cuadri}|p{3pt}@{}p{46.0\cuadri}}
		& &\\[-1\cuadri]
		\textbf{\color{color2}\Large #2}$\ $ & &\textbf{\Large #1}\\[-1\cuadri]
		& &\\[-0.8pt] \cline{2-3}
	\end{tabular}
}

\newcommand{\cajitaderecha}[7][]{%
	\ifthenelse{\isempty{#1}}{
		\cajitaalternante{
			\addcontentsline{toc}{section}{\numberline{\thesection} #2}
			\addtocontents{toc}{\protect\thispagestyle{empty}}
			\titulizador{#2}
			\begin{tabular}[b]{@{}p{17.5\cuadri}@{}p{1.5\cuadri}@{} x{34\cuadri}@{}}
				& &\\[0.5\cuadri]
				\begin{descripcion-cajita}
					\parskip 6pt\parindent 1em
					#3
				\end{descripcion-cajita}
				& &
				\begin{grafica-cajita}
				\begin{center}
					\ifthenelse{\isempty{#4}}{}{{\textbf{#4}}\\[-1pt]}
					\ifthenelse{\isempty{#5}}{
						$\ $\\[-0.1\cuadri]
					}{
						{\footnotesize\texttwelveudash$\,\,$#5$\,\,$\texttwelveudash}\\[0.6\cuadri]
					}
					#6
					\begin{flushleft}
						$\ $\\[-2\cuadri]
						\ \ \ \footnotesize Fuente: #7
					\end{flushleft}
				\end{center}
				\end{grafica-cajita}
			\end{tabular}
		}
	}{
		\cajitaalternante{
			\addcontentsline{toc}{section}{\numberline{\thesection} #1}
			\addtocontents{toc}{\protect\thispagestyle{empty}}
			\titulizador{#2}

			\begin{tabular}[b]{@{}p{17.5\cuadri}@{}p{1.5\cuadri}@{} x{34\cuadri}@{}}
				& &\\[0.5\cuadri]
				\begin{descripcion-cajita}
					\parskip 6pt\parindent 1em%
					#3
				\end{descripcion-cajita}
				& &
				\begin{grafica-cajita}
					\begin{center}
						\ifthenelse{\isempty{#4}}{}{{\textbf{#4}}\\[-1pt]}%
						\ifthenelse{\isempty{#5}}{
							$\ $\\[-0.1\cuadri]
						}{
							{\footnotesize\texttwelveudash$\,\,$#5$\,\,$\texttwelveudash}\\[0.6\cuadri]
						}
						#6
						\begin{flushleft}
							$\ $\\[-2\cuadri]
							\ \ \ \footnotesize Fuente: #7
						\end{flushleft}
					\end{center}
				\end{grafica-cajita}
			\end{tabular}
		}
	}
}

\newcommand{\cajitaizquierda}[7][]{
	\ifthenelse{\isempty{#1}}{
		\cajitaalternante{
			\addcontentsline{toc}{section}{\numberline{\thesection} #2}
			\addtocontents{toc}{\protect\thispagestyle{empty}}
			\titulizador{#2}
			\begin{tabular}[b]{@{}p{34\cuadri}@{}p{0\cuadri}@{} x{17.5\cuadri}@{}}
				& &\\[0.5\cuadri]
				\begin{grafica-cajita}
					\begin{center}
						\ifthenelse{\isempty{#4}}{}{{\textbf{#4}}\\[-1pt]}%
						\ifthenelse{\isempty{#5}}{
							$\ $\\[-0.1\cuadri]
						}{
							{\footnotesize\texttwelveudash$\,\,$#5$\,\,$\texttwelveudash}\\[0.6\cuadri]
						}
						#6
						\begin{flushleft}
							$\ $\\[-2\cuadri]
							\ \ \ \footnotesize Fuente: #7
						\end{flushleft}
					\end{center}
				\end{grafica-cajita}
				& &
				\begin{descripcion-cajita}
					\parskip 6pt\parindent 1em%
					#3
				\end{descripcion-cajita}
			\end{tabular}
		}
	}{
		\cajitaalternante{
			\addcontentsline{toc}{section}{\numberline{\thesection} #1}
			\addtocontents{toc}{\protect\thispagestyle{empty}}
			\titulizador{#2}
			\begin{tabular}[b]{@{}p{34\cuadri}@{}p{0\cuadri}@{} x{17.5\cuadri}@{}}
				& &\\[0.5\cuadri]
				\begin{grafica-cajita}
					\begin{center}
						\ifthenelse{\isempty{#4}}{}{{\textbf{#4}}\\[-1pt]}%
						\ifthenelse{\isempty{#5}}{
							$\ $\\[-0.1\cuadri]
						}{
							{\footnotesize\texttwelveudash$\,\,$#5$\,\,$\texttwelveudash}\\[0.6\cuadri]
						}
						#6
						\begin{flushleft}
							$\ $\\[-2\cuadri]
							\ \ \ \footnotesize Fuente: #7
						\end{flushleft}
					\end{center}
				\end{grafica-cajita}
				& &
				\begin{descripcion-cajita}
					\parskip 6pt\parindent 1em%
					#3
				\end{descripcion-cajita}
			\end{tabular}
		}
	}
}

\newcommand{\cajita}[7][]{
	\ifthenelse{\equal{\theupdown}{0}}{
		\cajitaizquierda[#1]{#2}{#3}{#4}{#5}{#6}{#7}
	}{
		\cajitaderecha[#1]{#2}{#3}{#4}{#5}{#6}{#7}
	}
}
%seguir formateando psudo-pep8
%%%%%%%%%%%%%%%  Cajita manual
\newcommand{\cajitaderechamanual}[8][]{%
\ifthenelse{\isempty{#1}}{
\cajitaalternante{
\addcontentsline{toc}{section}{\numberline{\thesection} #2}
\addtocontents{toc}{\protect\thispagestyle{empty}}
\titulizadormanual{#2}{#8}

\begin{tabular}[b]{@{}p{17.5\cuadri}@{}p{1.5\cuadri}@{} x{34\cuadri}@{}}
& &\\[0.5\cuadri]
\begin{descripcion-cajita}
\parskip 6pt\parindent 1em%
#3
\end{descripcion-cajita}
  & &
\begin{grafica-cajita}
\begin{center}
\ifthenelse{\isempty{#4}}{}{{\textbf{#4}}\\[-1pt]}%
\ifthenelse{\isempty{#5}}{$\ $\\[-0.1\cuadri]}{
	{\footnotesize\texttwelveudash$\,\,$#5$\,\,$\texttwelveudash}\\[0.6\cuadri]}
#6
\begin{flushleft}
$\ $\\[-2\cuadri]
\ \ \ \footnotesize Fuente: #7
\end{flushleft}
\end{center}
\end{grafica-cajita}
\end{tabular}
}}
{
\cajitaalternante{
\addcontentsline{toc}{section}{\numberline{\thesection} #1}
\addtocontents{toc}{\protect\thispagestyle{empty}}
\titulizadormanual{#2}{#8}

\begin{tabular}[b]{@{}p{17.5\cuadri}@{}p{1.5\cuadri}@{} x{34\cuadri}@{}}
& &\\[0.5\cuadri]
\begin{descripcion-cajita}
\parskip 6pt\parindent 1em%
#3
\end{descripcion-cajita}
  & &
\begin{grafica-cajita}
\begin{center}
\ifthenelse{\isempty{#4}}{}{{\textbf{#4}}\\[-1pt]}%
\ifthenelse{\isempty{#5}}{$\ $\\[-0.1\cuadri]}{
	{\footnotesize\texttwelveudash$\,\,$#5$\,\,$\texttwelveudash}\\[0.6\cuadri]}
#6
\begin{flushleft}
$\ $\\[-2\cuadri]
\ \ \ \footnotesize Fuente: #7
\end{flushleft}
\end{center}
\end{grafica-cajita}
\end{tabular}
}}
}

\newcommand{\cajitaizquierdamanual}[8][]{%
\ifthenelse{\isempty{#1}}{
\cajitaalternante{
\addcontentsline{toc}{section}{\numberline{\thesection} #2}
\addtocontents{toc}{\protect\thispagestyle{empty}}
\titulizadormanual{#2}{#8}

\begin{tabular}[b]{@{}p{34\cuadri}@{}p{0\cuadri}@{} x{17.5\cuadri}@{}}
& &\\[0.5\cuadri]
\begin{grafica-cajita}
\begin{center}
\ifthenelse{\isempty{#4}}{}{{\textbf{#4}}\\[-1pt]}%
\ifthenelse{\isempty{#5}}{$\ $\\[-0.1\cuadri]}{
	{\footnotesize\texttwelveudash$\,\,$#5$\,\,$\texttwelveudash}\\[0.6\cuadri]}
#6
\begin{flushleft}
$\ $\\[-2\cuadri]
\ \ \ \footnotesize Fuente: #7
\end{flushleft}
\end{center}
\end{grafica-cajita}
  & &
\begin{descripcion-cajita}
\parskip 6pt\parindent 1em%
#3
\end{descripcion-cajita}
\end{tabular}
}}
{
\cajitaalternante{
\addcontentsline{toc}{section}{\numberline{\thesection} #1}
\addtocontents{toc}{\protect\thispagestyle{empty}}
\titulizadormanual{#2}{#8}

\begin{tabular}[b]{@{}p{34\cuadri}@{}p{0\cuadri}@{} x{17.5\cuadri}@{}}
& &\\[0.5\cuadri]
\begin{grafica-cajita}
\begin{center}
\ifthenelse{\isempty{#4}}{}{{\textbf{#4}}\\[-1pt]}%
\ifthenelse{\isempty{#5}}{$\ $\\[-0.1\cuadri]}{
	{\footnotesize\texttwelveudash$\,\,$#5$\,\,$\texttwelveudash}\\[0.6\cuadri]}
#6
\begin{flushleft}
$\ $\\[-2\cuadri]
\ \ \ \footnotesize Fuente: #7
\end{flushleft}
\end{center}
\end{grafica-cajita}
  & &
\begin{descripcion-cajita}
\parskip 6pt\parindent 1em%
#3
\end{descripcion-cajita}
\end{tabular}
}}
}

\newcommand{\cajitamanual}[8][]{%
\ifthenelse{\equal{\theupdown}{0}}{
\cajitaizquierdamanual[#1]{#2}{#3}{#4}{#5}{#6}{#7}{#8}

}{%
\cajitaderechamanual[#1]{#2}{#3}{#4}{#5}{#6}{#7}{#8}

}
}

%%%%%%%%%%%%%%%  Cajita de tabla
\newcommand{\cajitaizquierdatabla}[7][]{%
	\ifthenelse{\isempty{#1}}{
		\cajitaalternante{
			\addcontentsline{toc}{section}{\numberline{\thesection} #2}
			\addtocontents{toc}{\protect\thispagestyle{empty}}
			\titulizador{#2}
			
			\begin{tabular}[b]{@{}p{34\cuadri}@{}p{0\cuadri}@{} x{17.5\cuadri}@{}}
				& &\\[0.5\cuadri]
				\begin{grafica-cajita}
					\begin{center}
						\ifthenelse{\isempty{#4}}{}{{\textbf{#4}}\\[-1pt]}%
						\ifthenelse{\isempty{#5}}{$\ $\\[-0.1\cuadri]}{
							{\footnotesize\texttwelveudash$\,\,$#5$\,\,$\texttwelveudash}\\[0.6\cuadri]}
						\ifthenelse{\isempty{#7}}{\renewcommand{\lafuente}{Estadísticas INE}}{\renewcommand{\lafuente}{#7}}%
						\begin{tabular}{l}
							#6\\[4mm]
							$\ $\\[-3.5mm]
							{\footnotesize $\ \ \ $Fuente: \lafuente}
						\end{tabular}
					\end{center}
				\end{grafica-cajita}
				& &
				\begin{descripcion-cajita}
					\parskip 6pt\parindent 1em%
					#3
				\end{descripcion-cajita}
			\end{tabular}
		}}
		{
			\cajitaalternante{
				\addcontentsline{toc}{section}{\numberline{\thesection} #1}
				\addtocontents{toc}{\protect\thispagestyle{empty}}
				\titulizador{#2}
				
				\begin{tabular}[b]{@{}p{34\cuadri}@{}p{0\cuadri}@{} x{17.5\cuadri}@{}}
					& &\\[0.5\cuadri]
					\begin{grafica-cajita}
						\begin{center}
							\ifthenelse{\isempty{#4}}{}{{\textbf{#4}}\\[-1pt]}%
							\ifthenelse{\isempty{#5}}{$\ $\\[-0.1\cuadri]}{
								{\footnotesize\texttwelveudash$\,\,$#5$\,\,$\texttwelveudash}\\[0.6\cuadri]}
							\ifthenelse{\isempty{#7}}{\renewcommand{\lafuente}{Estadísticas INE}}{\renewcommand{\lafuente}{#7}}%
							\begin{tabular}{l}
								#6\\[4mm]
								$\ $\\[-3.5mm]
								{\footnotesize $\ \ \ $Fuente: \lafuente}
							\end{tabular}
						\end{center}
					\end{grafica-cajita}
					& &
					\begin{descripcion-cajita}
						\parskip 6pt\parindent 1em%
						#3
					\end{descripcion-cajita}
				\end{tabular}
			}}
		}

\newcommand{\cajitaderechatabla}[7][]{%
	\ifthenelse{\isempty{#1}}{
		\cajitaalternante{
			\addcontentsline{toc}{section}{\numberline{\thesection} #2}
			\addtocontents{toc}{\protect\thispagestyle{empty}}
			\titulizador{#2}
			
			\begin{tabular}[b]{@{}p{17.5\cuadri}@{}p{1.5\cuadri}@{} x{34\cuadri}@{}}
				& &\\[0.5\cuadri]
				\begin{descripcion-cajita}
					\parskip 6pt\parindent 1em%
					#3
				\end{descripcion-cajita}
				& &
				\begin{grafica-cajita}
					\begin{center}
						\ifthenelse{\isempty{#4}}{}{{\textbf{#4}}\\[-1pt]}%
						\ifthenelse{\isempty{#5}}{$\ $\\[-0.1\cuadri]}{
							{\footnotesize\texttwelveudash$\,\,$#5$\,\,$\texttwelveudash}\\[0.6\cuadri]}
						\ifthenelse{\isempty{#7}}{\renewcommand{\lafuente}{Estadísticas INE}}{\renewcommand{\lafuente}{#7}}%
						\begin{tabular}{l}
							#6\\[4mm]
							$\ $\\[-3.5mm]
							{\footnotesize $\ \ \ $Fuente: \lafuente}
						\end{tabular}
					\end{center}
				\end{grafica-cajita}
			\end{tabular}
		}}
		{
			\cajitaalternante{
				\addcontentsline{toc}{section}{\numberline{\thesection} #1}
				\addtocontents{toc}{\protect\thispagestyle{empty}}
				\titulizador{#2}
				
				\begin{tabular}[b]{@{}p{17.5\cuadri}@{}p{1.5\cuadri}@{} x{34\cuadri}@{}}
					& &\\[0.5\cuadri]
					\begin{descripcion-cajita}
						\parskip 6pt\parindent 1em%
						#3
					\end{descripcion-cajita}
					& &
					\begin{grafica-cajita}
						\begin{center}
							\ifthenelse{\isempty{#4}}{}{{\textbf{#4}}\\[-1pt]}%
							\ifthenelse{\isempty{#5}}{$\ $\\[-0.1\cuadri]}{
								{\footnotesize\texttwelveudash$\,\,$#5$\,\,$\texttwelveudash}\\[0.6\cuadri]}
							\ifthenelse{\isempty{#7}}{\renewcommand{\lafuente}{Estadísticas INE}}{\renewcommand{\lafuente}{#7}}%
							\begin{tabular}{l}
								#6\\[4mm]
								$\ $\\[-3.5mm]
								{\footnotesize $\ \ \ $Fuente: \lafuente}
							\end{tabular}
						\end{center}
					\end{grafica-cajita}
				\end{tabular}
			}}
		}

\newcommand{\cajitatabla}[7][]{%
	\ifthenelse{\equal{\theupdown}{0}}{
		\cajitaizquierdatabla[#1]{#2}{#3}{#4}{#5}{#6}{#7}
		
		
	}{%
	\cajitaderechatabla[#1]{#2}{#3}{#4}{#5}{#6}{#7}
	
	
}
}

%%%%%%%%%%%%%%%% Macro de cajota

\newcommand{\cajota}[7][]{%
\noindent\begin{cajota-unica}\phantomsection\stepcounter{section}
\ifthenelse{\equal{\theupdown}{0}}{}{\setcounter{updown}{0}}
\ifthenelse{\isempty{#1}}{%
\addcontentsline{toc}{section}{\numberline{\thesection} #2}
\addtocontents{toc}{\protect\thispagestyle{empty}}
\titulizador{#2}}{%
\addcontentsline{toc}{section}{\numberline{\thesection} #1}
\addtocontents{toc}{\protect\thispagestyle{empty}}
\titulizador{#2}
}
\parskip 6pt\parindent 2em%
$\ $\\

#3$\ $\\[-1\cuadri]

\begin{center}
\ifthenelse{\isempty{#4}}{}{{\textbf{#4}}\\[-1pt]}%
\ifthenelse{\isempty{#5}}{$\ $\\[0.2\cuadri]}{
	{\footnotesize\texttwelveudash$\,\,$#5$\,\,$\texttwelveudash}\\[2.0\cuadri]}
#6
\begin{flushright}
$\ $\\[-1.5\cuadri]
\footnotesize Fuente: #7 $\ \ \ $ 
\end{flushright}
\end{center}
$\ $\\[-3\cuadri]

\end{cajota-unica}
}

%%%%%%%%% Cajota de Tabla

\newcommand{\cajotatabla}[7][]{%
	\noindent\begin{cajota-unica}\phantomsection\stepcounter{section}
		\ifthenelse{\equal{\theupdown}{0}}{}{\setcounter{updown}{0}}
		\ifthenelse{\isempty{#1}}{%
			\addcontentsline{toc}{section}{\numberline{\thesection} #2}
			\addtocontents{toc}{\protect\thispagestyle{empty}}
			\titulizador{#2}}{%
			\addcontentsline{toc}{section}{\numberline{\thesection} #1}
			\addtocontents{toc}{\protect\thispagestyle{empty}}
			\titulizador{#2}
		}
		\parskip 6pt\parindent 2em%
		$\ $\\
		
		#3$\ $\\[-1\cuadri]
		
		\begin{center}
			\ifthenelse{\isempty{#4}}{}{{\textbf{#4}}\\[-1pt]}%
			\ifthenelse{\isempty{#5}}{$\ $\\[0.2\cuadri]}{
			{\footnotesize\texttwelveudash$\,\,$#5$\,\,$\texttwelveudash}\\[2.0\cuadri]}
		\ifthenelse{\isempty{#7}}{\renewcommand{\lafuente}{Estadísticas INE}}{\renewcommand{\lafuente}{#7}}%
			\begin{tabular}{l}
				#6\\[4mm]
				$\ $\\[-3.5mm]
				{\footnotesize $\ \ \ $Fuente: \lafuente}
			\end{tabular}
		\end{center}
		$\ $\\[-3\cuadri]
		
	\end{cajota-unica}
	
	
}

%%%%%%%%%% Macro de capítulo

%previos

\newcommand{\capituloencabezado}{}

\newcommand{\capitulocondescripcion}[2]{%
\begin{fondo-capitulo}
$\ $\\[7.00\cuadri]
\begin{tabular}{p{1.28in}p{1.2in}p{4.75in}}
 & \begin{numcapitulo}\fontsize{0.875in}{1em}\selectfont\color{white}\centering\textbf{\thechapter}\end{numcapitulo} & \fontsize{0.4in}{3em}\selectfont \Bold \begin{tabular}[t]{p{4.75in}} #1 
 \end{tabular} \\ 
\end{tabular}\\[1.75in]

\begin{tabular}{p{2.5in}p{4.9in}}
 &\parskip 2.5ex \parindent 2em \Large #2
 
\\ 
\end{tabular} 
\end{fondo-capitulo}
}

\newcommand{\capitulosindescripcion}[1]{%
\begin{fondo-capitulo-no-descripcion}
$\ $\\[22.65\cuadri]
\begin{tabular}{p{1.28in}p{1.2in}p{4.75in}}
 & \begin{numcapitulo}\fontsize{0.875in}{1em}\selectfont\color{white}\centering\textbf{\thechapter}\end{numcapitulo} & \fontsize{0.4in}{3em}\selectfont \Bold \begin{tabular}[t]{p{4.75in}} #1 
 \end{tabular} \\ 
\end{tabular}
\end{fondo-capitulo-no-descripcion}
}

\newcommand{\rayitatoc}{\addtocontents{toc}{\protect\addvspace{0.2\baselineskip}{\color{color2}\hrule height 0.9pt} \addvspace{0.6\baselineskip} \color{black}} }

% El macro definitivo

\newcommand{\partes}{}

\newcommand{\INEchaptercarta}[3][]{%
	\cleardoublepage\stepcounter{chapter}\addtocontents{toc}{\protect\addvspace{0.6\baselineskip}\color{color1}}%
	\phantomsection
	\ifthenelse{\isempty{#1}}{%
		\addcontentsline{toc}{chapter}{\numberline{\thechapter}#2}%
		\renewcommand{\capituloencabezado}{#2}%
	}{
	\addcontentsline{toc}{chapter}{\numberline{\thechapter}#1}%
	\renewcommand{\capituloencabezado}{#1}%
}
\thispagestyle{empty}%

\ifthenelse{\equal{\partes}{}}{\rayitatoc}{} %
\addtocontents{toc}{\color{black}\addvspace{0.2\baselineskip}}
\ifthenelse{\equal{\unexpanded{#3}}{}}{%
	\capitulosindescripcion{#2}
}{%
\capitulocondescripcion{#2}{#3}
}

\cleardoublepage
\setcounter{section}{0}
\setcounter{updown}{0}
\setcounter{footnote}{0}
}

%%%%%%%%%%  PARTES

\newcommand{\partesindescripcion}[1]{%
\begin{fondo-parte}
$\ $\\[17.65\cuadri]
\begin{tabular}{p{1.4in}p{5.75in}}
 & \begin{tabular}[t]{x{5.75in}}
\Cinzel \fontsize{0.4in}{3.5em}\selectfont \textbf{PARTE \Roman{parte}}\\[0.4in] \hline
\\[-0.08in]
\Cinzel\fontsize{0.4in}{3.5em}\selectfont {\color{color1!70!black} #1}  \\[0.2in]\hline
 \end{tabular} \\ 
\end{tabular}
\end{fondo-parte}
}

\newcommand{\partecondescripcion}[2]{%
\begin{fondo-parte}
$\ $\\[7.00\cuadri]
\begin{tabular}{p{1.4in}p{5.75in}}
 &  \begin{tabular}[t]{x{5.75in}} 
\Cinzel\fontsize{0.4in}{3.5em}\selectfont \textbf{PARTE \Roman{parte}}\\[0.4in] \hline
\\[-0.08in]
\Cinzel\fontsize{0.4in}{3.5em}\selectfont {\color{color1!70!black} #1} 
  \\[0.2in]\hline
 \end{tabular} \\ 
\end{tabular}\\[0.85in]

\begin{tabular}{p{1.4in}p{5.75in}}
 &\parskip 2.5ex \parindent 2em \Logos\LARGE
 \begin{multicols}{2}
 #2
\end{multicols}
\\ 
\end{tabular} 
\end{fondo-parte}
}

% El macro de parte definitivo

\newcounter{parte}
\setcounter{parte}{0}

\newcommand{\INEpartecarta}[3][]{%
	\cleardoublepage\stepcounter{parte}\addtocontents{toc}{\protect\addvspace{3.1\baselineskip}\color{color2!60!black}}%
	\phantomsection
	\ifthenelse{\isempty{#1}}{
	\addtocontents{toc}{\protect \textbf{\Cinzel\large PARTE {\Roman{parte}} }\\[1.0mm] {\Cinzel\large #2 }\\[-2.5mm]
			\protect \par} \rayitatoc	
		}{
\addtocontents{toc}{\protect \textbf{\Cinzel\large PARTE {\Roman{parte}} }\\[1.0mm] {\Cinzel\large #1 }\\[-2.5mm]
	\protect \par} \rayitatoc
}
\thispagestyle{empty}%
\addtocontents{toc}{\protect\addvspace{-0.1\baselineskip}\color{black}}%
\ifthenelse{\equal{\unexpanded{#3}}{}}{%
	\partesindescripcion{#2}
}{%
\partecondescripcion{#2}{#3}
}

\cleardoublepage
\setcounter{section}{0}
\setcounter{updown}{0}
}

%%%%%%%%

\let\oldappendix\appendix

\renewcommand{\appendix}{
\cleardoublepage
\oldappendix

$\ $
\vspace{6.6cm}

\thispagestyle{empty}
\begin{center}
	\fontsize{16mm}{1em}\selectfont\Bold \color{color2!80!black} APÉNDICES
\end{center}
\addtocontents{toc}{\protect\addvspace{0.6\baselineskip}}
\addcontentsline{toc}{chapter}{APÉNDICES}
\cleardoublepage
\renewcommand{\rayitatoc}{}
}

% % % % % % % % % % % % % % % % % % % % % % % % % % % %
%  Parte en construcción

% De la ENEI vieja.
\newtcolorbox{fondo}{width=6.5in, height=9in, nobeforeafter, boxrule=0pt, colframe=white, left=-3pt, bottom=-1in, top=-13pt, right=0pt, arc=0pt, enlarge bottom by= -3in, colback= white}
\newcommand{\hoja}[1]{\noindent\begin{fondo} #1 \end{fondo}\clearpage}


\newcommand{\titulo}[1]{
$\ $\\[0.3in]
\noindent{\color{color1!80!black}\LARGE \textbf{#1}}\\[-0.1in]
{\color{color1}\hrule}
$\ $\\[-0.1in]
}