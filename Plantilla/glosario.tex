{\setlength{\parindent}{0cm} 
\textbf{Bienes de consumo:} Bienes o servicios generalmente destinados al consumo final, no a un proceso productivo.

\textbf{Bienes no transables:} Bienes que por su naturaleza no son susceptibles de ser comercializados en el mercado internacional, por lo que su precio se determina por las condiciones de oferta y demanda en el mercado interno.

\textbf{Bienes transables:} Bienes susceptibles de ser comercializados internacionalmente (exportados o importados). Su precio tenderá a reflejar el precio internacional más aranceles y costos de transporte en moneda nacional.

\textbf{Consumo:} Actividad que consiste en el uso de bienes y servicios para la satisfacción de las necesidades o deseos humanos individuales o colectivos.

\textbf{Deflación:} Consiste en el descenso general y continuo de precios, lo que origina una disminución en el ritmo de la actividad económica; afecta, entre otros aspectos, al empleo y a la producción de bienes y servicios de un país.

\textbf{División de gasto:} Agregación que estructura el cálculo del Índice de Precios al Consumidor -IPC-. Las divisiones actuales son: alimentos y bebidas no alcohólicas; bebidas alcohólicas y tabaco; prendas de vestir y calzado; vivienda, agua, electricidad, gas y otros combustibles; muebles y artículos para el hogar; salud; transporte, comunicaciones; recreación y cultura; educación; restaurantes y hoteles; y bienes y servicios diversos.

\textbf{Impacto o incidencia:} Influencia que ejerce cada gasto básico, división de gasto o región en el porcentaje de variación del índice de precios general. La suma de todas las incidencias que integran la estructura del IPC es igual a la variación total del IPC en el mes de estudio.

\textbf{Índice:} Número que expresa la evolución en el tiempo de los valores de una variable o magnitud, tales como precios, cotizaciones, desempleo, entre otros. Los índices están referidos a una fecha base a la cual se le asigna arbitrariamente un valor que por lo general es 100.

\textbf{Índice de Precios al Consumidor (IPC):} Mide la evolución de la variación de los precios medios de los bienes y servicios de una canasta representativa del consumo de los hogares de una determinada área geográfica, con referencia a un período de tiempo.

\textbf{Inflación:} Alza generalizada y persistente en el nivel de los precios internos de la economía del país.

\textbf{Inflación importada:} Efecto de la evolución de los precios internacionales en la inflación interna. Incluye bienes de consumo cuyos costos dependen significativamente de los precios de commodities (pan, fideos, aceites y combustibles) y aquellos que son en gran parte o enteramente de origen importado (vehículos, combustibles, aparatos electrodomésticos y medicinas).

\textbf{Organización de las Naciones Unidas para la Alimentación y la Agricultura (FAO por sus siglas en inglés):} Entidad de Naciones Unidas, cuyo mandato consiste en “aumentar la productividad agrícola, elevar el nivel de vida de la población rural y contribuir al crecimiento de la economía mundial”.

\textbf{Período de referencia:} Aquel en el que se realiza la recolección de precios del conjunto de bienes y servicios a los que se dará seguimiento para el cálculo de un índice.

\textbf{Petróleo o crudo:} Mezcla de hidrocarburos que existe en forma líquida en reservorios del subsuelo y tiene un punto de inflamación menor a 65,6 C°. El crudo es la materia prima que se refina en gasolina, aceite combustible, propano, petroquímicos y otros productos. En Guatemala se utiliza como referencia el precio del West Texas Intermediate (WTI) que es un promedio, en cuanto a calidad, del petróleo producido en los campos occidentales del estado de Texas (Estados Unidos). Se emplea como precio de referencia para fijar el precio de otros tipos de petróleo producidos en medio oriente.

\textbf{Poder adquisitivo:} Con relación a personas, se refiere a la capacidad económica para adquirir bienes o servicios. Respecto al dinero, representa la cantidad de bienes y servicios que se adquieren por una suma determinada respecto a la cantidad que se habría adquirido anteriormente en un período base. En el ámbito nacional, el poder adquisitivo se verá afectado por la inflación y, en el ámbito internacional, por el tipo de cambio en la moneda nacional respecto a las monedas de otros países.

\textbf{Ponderación:} Importancia relativa del gasto de los hogares de un producto, en el total del gasto de los bienes y servicios reportados en la Encuesta de Ingresos y Gastos Familiares (ENIGFAM 2009 – 2010).

\textbf{Precio:} Valoración de un bien o servicio en unidades monetarias u otro instrumento de cambio. El precio puede ser fijado libremente por el mercado en función de la oferta y la demanda, o por las autoridades, en cuyo caso se trataría de un precio controlado.

\textbf{Puntos Básicos:} Centésimo de punto porcentual (0.01\%). 100 puntos básicos = 1\%. Se utilizan principalmente para expresar las diferencias de tasa o rendimiento.

\textbf{Región:} Guatemala está dividida en ocho regiones: la Región I (metropolitana); la Región II (norte); la Región III (Nor-Oriente); la Región IV (Sur-Oriente); la Región V (central); la Región VI (Sur-Occidente); la Región VII (Nor-Occidente); y la Región VIII (Petén).

\textbf{Tasa de interés:} Precio que se paga por el uso del dinero. Suele expresarse en términos porcentuales y referirse a un período de un año.

\textbf{Tipo de cambio:} Relación entre dos divisas es la tasa o relación de proporción que existe entre el valor de una y la otra. Dicha tasa es un indicador que expresa cuántas unidades de una divisa se necesitan para obtener una unidad de la otra.

\textbf{Variación acumulada:} Se refiere al cambio relativo del IPC del mes en estudio con respecto a diciembre del año anterior.

\textbf{Variación interanual:} Se refiere al cambio relativo del IPC del mes en estudio con respecto al mismo mes del año anterior

\textbf{Variación mensual:} Se refiere al cambio relativo del IPC de un mes en estudio con respecto al mes anterior.
}