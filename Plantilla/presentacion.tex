La Encuesta Nacional de Condiciones de Vida \textemdash Encovi\textemdash, tiene como principal objetivo, conocer y evaluar las condiciones de vida de la población, así como determinar los niveles de pobreza existentes en Guatemala y los factores que los determinan.
 
 La Encovi adopta la metodología de las encuestas de condiciones de vida, que en lo fundamental, combinan aspectos cuantitativos y cualitativos mediante la aplicación de un conjunto integrado de formularios sobre la calidad de vida de los hogares y las personas. Esta perspectiva permite una mejor aproximación a los diferentes aspectos y componentes de la pobreza, es decir, a su carácter multidimensional. Permite además, abordar el estudio de la desigualdad y la identificación de mecanismos de intervención eficaz que promuevan mejoras sustantivas de las condiciones de vida.

Debido a la amplitud de los temas investigados en la Encovi, el informe de resultados de esta encuesta se presentará en tres tomos en el 2016. Sin embargo, para que la población en general tenga oportunamente    los principales resultados de la Encovi 2014, se presenta este informe, el cúal es un extracto del documento de mayor volumen que se publicará el próximo año.  El presente documento tiene tres capítulos: Pobreza, Desigualdad y Objetivos de Desarrollo del Milenio, los que permiten visualizar la evolución de las condiciones de vida del país en los últimos quince años. 


El Instituto Nacional de Estadística agradece a todas las instituciones públicas, privadas y de cooperación internacional que apoyaron este esfuerzo. Asimismo, a los hogares que abrieron sus puertas y  brindaron  la información  solicitada, sin la cual no hubiera sido posible  la elaboración de este informe.\\[1cm] 