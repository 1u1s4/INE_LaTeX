El presente informe contiene los principales resultados del Índice de Precios al Consumidor (IPC) generado por el Instituto
Nacional de Estadística -INE-. Como indicador macroeconómico, este se utiliza para medir el comportamiento del nivel
general de precios de la economía del país, tomando como base los precios observados en el mes de referencia.

Las variaciones del Índice de Precios al Consumidor más importantes a Diciembre de 2015 son las siguientes: se registró
una variación intermensual de 0.43\%, una variación interanual de 3.07\% y una variación acumulada de 3.07\%.

Para profundizar en los resultados, el informe contiene diez capítulos; en el primero se presenta información sobre variables
que influyen en el nivel de precios internos, en el segundo se encuentran los resultados del IPC a nivel nacional,
y en los restantes se detalla la información del indicador de cada una de las ocho regiones del país. Al final se incluye un
glosario que contiene la definición de los principales conceptos relacionados con el IPC y la metodología de cálculo para la
obtención de los diferentes índices y variaciones.

\begin{center}
\textbf{Hugo Allan Garcia}\\
Gerente\\
Instituto Nacional de Estadística
\end{center}