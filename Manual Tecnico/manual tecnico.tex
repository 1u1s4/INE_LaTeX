\documentclass[12pt]{article}
\usepackage{graphicx} % This lets you include figures
\usepackage[rightcaption]{sidecap}
\usepackage{subcaption}
\usepackage{wrapfig}
\usepackage{float}
\usepackage[spanish]{babel}
\languageshorthands{spanish}
\usepackage{imakeidx}
\usepackage{cite}
\makeindex
\usepackage{hyperref}
\usepackage{hologo}


\title{Manual Técnico \\ Carta3}
\author{Luis A. Alvarado}
\date{\today}
\begin{document}
\maketitle{}

\tableofcontents

\clearpage
\newpage

\section{Paquetes}
\subsection{geometry}
Proporciona una interfaz de usuario fácil y flexible para personalizar el diseño de la página, implementando mecanismos de autocentrado y autoequilibrado para que los usuarios solo tengan que dar la mínima descripción del diseño de la página.\cite{geometry}
\subsection{amsmath}
Proporciona un conjunto de opciones para mostrar ecuaciones. \cite{amsmath}
\subsection{amsfonts}
Colección amplia de fuentes para usar en matemáticas. \cite{amsfonts}
\subsection{amssymb}
Ofrece toneladas de símbolos matemáticos, como flechas, operadores, caracteres especiales, figuras geométricas, etc. \cite{amssymb}
\subsection{graphicx}
Se basa en el paquete \textit{graphics} y proporciona una interfaz clave-valor para argumentos opcionales del comando \textit{\textbackslash includegraphics}. Esta interfaz proporciona funciones que van mucho más allá de lo que ofrece el paquete de \textit{graphics} por sí solo. \cite{graphicx}
\subsection{pdfpages}
Simplifica la inclusión de documentos PDF externos de varias páginas en documentos \LaTeX.\cite{pdfpages}
\subsection{setspace}
Agrega soporte para establecer el espacio entre líneas en un documento. Las opciones de paquete incluyen espacio simple, espacio medio y espacio doble. \cite{setspace}
\subsection{xltxtra}
Se utiliza para proporcionar una serie de funciones que son útiles para la composición tipográfica de documentos con \hologo{XeLaTeX}. \cite{xltxtra}
\subsection{enumitem}
Permite el control sobre el diseño de los tres entornos de lista básicos: \textit{enumerate}, \textit{itemize} y \textit{description}.\cite{enumitem}
\subsection{xifthen}
Este paquete amplía el paquete \textit{ifthen} al implementar nuevos comandos para ir dentro del primer argumento de \textit{\textbackslash ifthenelse} para probar si una cadena es nula o no, si un comando está definido o es equivalente a otro. El paquete también permite el uso de expresiones complejas tal como las introdujo el paquete \textit{calc}, junto con la capacidad de definir nuevos comandos para manejar pruebas complejas.\cite{xifthen}
\subsection{xargs}
Agrega versiones extendidas de \textit{\textbackslash newcommand} y comandos \hologo{LaTeX} relacionados, que permiten una definición fácil y robusta de macros con muchos argumentos opcionales, usando una sintaxis clara.\cite{xargs}
\subsection{booktabs}
Mejora la calidad de las tablas en \hologo{LaTeX}, brindando comandos adicionales y optimización entre bastidores.\cite{booktabs}
\subsection{multirow}
Agrega mucha flexibilidad, incluida una opción para especificar una entrada en el ancho "natural" de su texto. Cree celdas tabulares que abarquen varias filas.\cite{multirow}
\subsection{pdflscape}
El paquete agrega compatibilidad con PDF al entorno horizontal del paquete \textit{lscape}, configurando el atributo PDF \textit{/Rotate}.\cite{pdflscape}
\subsection{rotating}
Un paquete construido sobre el paquete de \hologo{LaTeX} \textit{graphics} estándar para realizar todos los diferentes tipos de rotación que uno pueda desear, incluyendo figuras y tablas completas con sus leyendas.\cite{rotating}
\subsection{bigstrut}
\textit{\textbackslash bigstrut} produce un strut que es \textit{\textbackslash bigstrutjot} más alto, más bajo o ambos, que el strut de matriz/tabla estándar. Úselo en las entradas de la tabla que están adyacentes a una línea \textit{\textbackslash h}, para dejar un poco de espacio adicional.\cite{bigstrut}
\subsection{longtable}
Permite escribir tablas que continúan en la página siguiente.\cite{longtable}
\subsection{fancyhdr}
Proporciona amplias funciones, tanto para construir encabezados y pies de página, como para controlar su uso.\cite{fancyhdr}
\subsection{url}
El comando \textit{\textbackslash url} es una forma de comando textual que permite saltos de línea en ciertos caracteres o combinaciones de caracteres, acepta la reconfiguración y, por lo general, se puede usar en el argumento de otro comando. El comando está diseñado para direcciones de correo electrónico, enlaces de hipertexto, directorios/rutas, etc., que normalmente no tienen espacios, por lo que al por defecto, el paquete ignora los espacios en su argumento.\cite{url}
% aqui sigue
\subsection{tikz}
PGF es un paquete de macros para crear gráficos. Es independiente de la plataforma y el formato y funciona junto con los controladores back-end de \hologo{TeX} más importantes, incluidos \hologo{pdfTeX} y dvips. Viene con una capa de sintaxis fácil de usar llamada TikZ.\cite{tikz}
\subsubsection{TikZ Library calc}
Permite cálculos de coordenadas avanzados.\cite{calc}
\subsubsection{TikZ Library positioning}
Permite colocar nodos en una dirección y distancia específicas de otros nodos.\cite{positioning}
\subsection{array}
Una implementación extendida de los entornos \textit{array} y \textit{tabular}, que amplía las opciones para los formatos de columna y proporciona especificaciones de formato "programable".\cite{array}
\subsection{tocloft}
Proporciona medios para controlar el diseño tipográfico de la Tabla de contenido, la Lista de figuras y la Lista de tablas.\cite{tocloft}
\subsection{hyperref}
Se usa para manejar comandos de referencias cruzadas en \hologo{LaTeX} para producir enlaces de hipertexto en el documento.\cite{hyperref}
\subsubsection{hidelinks}
Oculta los enlaces (quitar color y borde).\cite{hidelinks}
\subsubsection{verbose}
Se imprimen mensajes de diagnóstico adicionales en el archivo de registro.\cite{verbose}
\subsection{colortbl}
Permite colorear filas y columnas, e incluso celdas individuales.\cite{colortbl}
\subsection{multicol}
Define un entorno de varias columnas que compone el texto en varias columnas (hasta un máximo de 10) y (de forma predeterminada) equilibra el final de cada columna al final del entorno.\cite{multicol}
\subsection{changepage}
Proporciona comandos para cambiar el diseño de la página en medio de un documento y para verificar de manera sólida la composición tipográfica en páginas pares o impares.\cite{changepage}
\subsection{tcolorbox}
Proporciona un entorno para cuadros de texto coloreados y enmarcados con una línea de título.\cite{tcolorbox}
\subsubsection{skins}
Es una librería de tcolorbox que permite modificar el aspecto de la caja, como el color, degradado, estilo del titulo y muchas otras características. Para una mejor comprensión de esta librería ver la sección 10 de la cita.\cite{tcolorbox_lib}
\subsubsection{breakable}
Admite la ruptura automática de un tcolorbox. Para una mejor comprensión de esta librería ver la sección 19 de la cita. \cite{tcolorbox_lib}
\subsubsection{hooks}
Es un marcador de posición en algún código \hologo{LaTeX} donde se puede agregar código adicional. Para una mejor comprensión de esta librería ver la sección 23 de la cita. \cite{tcolorbox_lib}
\subsection{siunitx}
Tiene como objetivo proporcionar un método unificado para que los usuarios de \hologo{LaTeX} escriban números y unidades de forma correcta y sencilla.\cite{siunitx}
\subsubsection{input-decimal-markers}
Permite seleccionar el separador decimal entre coma (,) o punto (.).\cite{siunitx_param}
\subsubsection{input-ignore}
Ignora los tokens de esta lista.\cite{siunitx_param}
\subsubsection{group-separator}
Permite seleccionar el separador de unidades, decenas, centenas, etc.\cite{siunitx_param}
\subsection{polyglossia}
Este paquete proporciona un reemplazo completo de Babel, este paquete administra reglas tipográficas (y otras) determinadas culturalmente para una amplia gama de idiomas\cite{babel}, para usuarios de \hologo{LuaLaTeX} y \hologo{XeLaTeX}.\cite{polyglossia}



\newpage
\bibliographystyle{unsrt}
\bibliography{bibliografia}
\end{document}

\subsection{•}
\cite{•}